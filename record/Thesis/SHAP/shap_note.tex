\documentclass[a4paper,12pt]{article}
\usepackage{ctex}         % 支持中文
\usepackage{amsmath, amssymb, amsfonts} % 数学公式
\usepackage{graphicx}     % 插入图片
\usepackage{hyperref}     % 生成超链接
\usepackage{geometry}     % 页面布局
\usepackage{listings}     % 代码块
\usepackage{xcolor}       % 颜色支持
\geometry{a4paper, margin=1in} % 页边距
\hypersetup{colorlinks=true, linkcolor=blue, urlcolor=cyan}

% 代码样式
\lstset{
    language=Python, % 代码语言
    basicstyle=\ttfamily\footnotesize,
    keywordstyle=\color{blue},
    commentstyle=\color{gray},
    stringstyle=\color{red},
    breaklines=true,
    backgroundcolor=\color{gray!10},
    frame=single
}

\title{论文阅读笔记}
\author{你的姓名}
\date{\today}

\begin{document}

\maketitle

\begin{abstract}
本文是对论文《XXX》的学习笔记,主要总结了其研究背景、方法、实验结果,并对其优缺点进行了分析。
\end{abstract}

\textbf{关键词:} 机器学习、XAI、Shapley值、模型解释性

\section{论文基本信息}
\textbf{论文标题:} XXX \\
\textbf{作者:} XXX 等 \\
\textbf{发表会议/期刊:} NeurIPS 2023 \\
\textbf{链接:} \href{https://arxiv.org/abs/XXX}{arXiv链接}

\section{研究背景与问题}
本论文的研究目标是解决 \textbf{XXX} 问题。现有方法存在如下问题:
\begin{itemize}
    \item 方法 A 在 XXX 方面表现不佳;
    \item 方法 B 计算复杂度高,难以应用于大规模数据;
    \item 方法 C 解释性较差,无法有效解释模型决策。
\end{itemize}
本文提出了一种新方法 **XXX**,通过 XXX 改进了现有方法的局限性。

\section{方法介绍}
本文方法主要包含以下几个部分:

\subsection{数学公式}
本文基于 Shapley 值计算特征贡献,其计算公式如下:
\begin{equation}
\phi_i = \sum_{S \subseteq F \setminus \{i\}} \frac{|S|!(|F| - |S| - 1)!}{|F|!} \left[ f_{S \cup \{i\}}(x) - f_S(x) \right]
\end{equation}
其中:
- \( \phi_i \) 表示特征 \( x_i \) 的归因值;
- \( S \) 是特征子集;
- \( f_S(x) \) 是仅包含 \( S \) 特征的模型预测值。

\subsection{算法流程}
论文提出的方法 **XXX** 主要包括以下步骤:
\begin{enumerate}
    \item 读取输入数据 \( x \) 并进行预处理;
    \item 计算特征子集 \( S \) 对预测结果的影响;
    \item 使用 Shapley 值计算每个特征的重要性;
    \item 生成最终的模型解释。
\end{enumerate}

\subsection{代码示例}
论文中的方法可以用 Python 实现,代码示例如下:
\begin{lstlisting}
import shap
import xgboost
\end{lstlisting}

\section{实验与结果分析}
实验部分使用数据集 **XXX**,并与现有方法进行对比。

\begin{figure}[h]
    \centering
    %\includegraphics[width=0.8\textwidth]{}
    \caption{实验结果对比}
    \label{fig:results}
\end{figure}

从图 \ref{fig:results} 可以看出,新方法 **XXX** 在指标 **AUC** 和 **解释性** 方面优于现有方法。

\section{优缺点分析}
\subsection{优点}
\begin{itemize}
    \item 计算效率高,相比传统方法减少了 50\% 计算时间;
    \item 可解释性强,能够有效解释模型预测;
    \item 适用于多种模型,如 XGBoost、深度学习等。
\end{itemize}

\subsection{缺点}
\begin{itemize}
    \item 依赖于特定的先验假设,可能在部分任务上表现不佳;
    \item 需要大量计算 Shapley 值,计算复杂度仍较高。
\end{itemize}

\section{个人总结}
本论文提出了一种新的 **XXX 方法**,在 **解释性 AI** 领域具有重要意义。相比于 LIME 和 DeepLIFT,SHAP 在理论上更加稳健,并能够更合理地分配特征贡献。然而,该方法仍存在计算成本高的问题,未来可以考虑使用 **近似方法**(如 Kernel SHAP)来提高计算效率。此外,论文中的实验虽然全面,但缺乏真实业务场景的应用,后续可以尝试在 **金融欺诈检测** 等实际问题上进行实验。

\section{参考文献}
\begin{thebibliography}{99}

\end{thebibliography}

\end{document}